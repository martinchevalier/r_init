\documentclass[12pt,ignorenonframetext,]{beamer}
\setbeamertemplate{caption}[numbered]
\setbeamertemplate{caption label separator}{: }
\setbeamercolor{caption name}{fg=normal text.fg}
\beamertemplatenavigationsymbolsempty
\usepackage{lmodern}
\usepackage{amssymb,amsmath}
\usepackage{ifxetex,ifluatex}
\usepackage{fixltx2e} % provides \textsubscript
\ifnum 0\ifxetex 1\fi\ifluatex 1\fi=0 % if pdftex
  \usepackage[T1]{fontenc}
  \usepackage[utf8]{inputenc}
\else % if luatex or xelatex
  \ifxetex
    \usepackage{mathspec}
  \else
    \usepackage{fontspec}
  \fi
  \defaultfontfeatures{Ligatures=TeX,Scale=MatchLowercase}
\fi
% use upquote if available, for straight quotes in verbatim environments
\IfFileExists{upquote.sty}{\usepackage{upquote}}{}
% use microtype if available
\IfFileExists{microtype.sty}{%
\usepackage{microtype}
\UseMicrotypeSet[protrusion]{basicmath} % disable protrusion for tt fonts
}{}
\newif\ifbibliography
\hypersetup{
            pdftitle={Formation R initiation},
            pdfborder={0 0 0},
            breaklinks=true}
\urlstyle{same}  % don't use monospace font for urls

% Prevent slide breaks in the middle of a paragraph:
\widowpenalties 1 10000
\raggedbottom

\AtBeginPart{
  \let\insertpartnumber\relax
  \let\partname\relax
  \frame{\partpage}
}
\AtBeginSection{
  \ifbibliography
  \else
    \let\insertsectionnumber\relax
    \let\sectionname\relax
    \frame{\sectionpage}
  \fi
}
\AtBeginSubsection{
  \let\insertsubsectionnumber\relax
  \let\subsectionname\relax
  \frame{\subsectionpage}
}

\setlength{\parindent}{0pt}
\setlength{\parskip}{6pt plus 2pt minus 1pt}
\setlength{\emergencystretch}{3em}  % prevent overfull lines
\providecommand{\tightlist}{%
  \setlength{\itemsep}{0pt}\setlength{\parskip}{0pt}}
\setcounter{secnumdepth}{0}
% Packages à charger
\usepackage[french]{babel}
\usepackage{lmodern}
\usepackage{graphicx}
\usepackage{xcolor}
\usepackage{textcomp} 
\usepackage{amsmath, amsfonts, amssymb, amsthm}
\usepackage{booktabs,multirow}
\usepackage{setspace}
\usepackage{float}
\usepackage{pgfpages}
\usepackage{colortbl}
\usepackage{epstopdf}
\usepackage{framed}
\usepackage{etoolbox}
\usepackage{tikz}
\usepackage{csquotes}


% Définition de couleurs utilisées par la suite
\definecolor{shadecolor}{RGB}{248,248,248}
\definecolor{grayInsee}{HTML}{5a5758}
\definecolor{redInsee}{HTML}{ed1443}

% Personnalisation du thème du beamer
\usetheme{default}
\setbeamertemplate{navigation symbols}{}
\setbeamertemplate{footline}[frame number]
\setbeamercolor{title}{fg=grayInsee}
\setbeamercolor{section in toc}{fg=redInsee}
\setbeamercolor{subsection in toc}{fg=grayInsee}
\setbeamertemplate{frametitle}{%
	\large \textcolor{grayInsee}{\subsecname}
	\\ \vspace{0.1cm} \Large \textcolor{redInsee}{\insertframetitle}
}
\setbeamercolor{local structure}{fg=redInsee}

% Instruction spécifiques à tikz
\usetikzlibrary{shapes,arrows,calc, positioning}
\tikzstyle{input} = [draw, rectangle,rounded corners, text width=2.5cm, fill=green!20, node distance=0.5cm, minimum height=2em, text centered]
\tikzstyle{output} = [draw, ellipse,fill=red!20, node distance=0.5cm, minimum height=2em, text centered]
\tikzstyle{block} = [rectangle, draw, fill=blue!20, 
    text width=1.5cm, text centered, minimum height=2em, node distance = 0.5cm]
\tikzstyle{line} = [draw, -latex', shorten >=2pt, shorten <=2pt]
\tikzset{
  invisible/.style={opacity=0},
  visible on/.style={alt={#1{}{invisible}}},
  alt/.code args={<#1>#2#3}{%
    \alt<#1>{\pgfkeysalso{#2}}{\pgfkeysalso{#3}} % \pgfkeysalso doesn't change the path
  },
}

% Personnalisation des débuts de partie et de sous-partie
\AtBeginSection[]
{\ifnum \thesection>0
  \begin{frame}
  \vfill
  \begin{center}
  \LARGE
  \textcolor{grayInsee}{\insertsectionhead}
  \end{center}
  \vfill
  \end{frame}
\else
\fi
\subsection*{\secname}
}
\AtBeginSubsection[]{}

% Affichage d'un fond gris derrière les exemples de code
\ifcsmacro{Shaded}{
  \renewenvironment{Shaded}{\begin{snugshade}}{\end{snugshade}}
}

% Page de garde
\institute{
	\includegraphics[height = 2.5cm]{Logo_Insee.png} \\ ~ \\ 
	\normalsize Martin \textsc{Chevalier} (Insee)
}
\author{22 et 23 janvier 2018}
\date{}

% Commande outil pour les exemples, etc.
\newcommand{\aparte}[2]{
	{\small\textsf{\textcolor{redInsee}{\textbf{#1}} #2}}
}

\newcommand{\strong}[1]{\textbf{\textcolor{redInsee}{#1}}}

\newcommand{\link}[1]{\textcolor{redInsee}{\underline{#1}}}

\title{Formation R initiation}
\date{}

\begin{document}
\frame{\titlepage}

\subsection*{Formation R initiation}\label{formation-r-initiation}
\addcontentsline{toc}{subsection}{Formation R initiation}

\begin{frame}{Objectifs et pédagogie}

\strong{Objectifs de la formation}

\begin{enumerate}
\def\labelenumi{\arabic{enumi}.}
\item
  \pause Acquérir des points de repères et des réflexes dans
  l'utilisation du logiciel R;
\item
  \pause Savoir travailler de façon autonome sur des données
  statistiques dans une perspective Insee;
\item
  \pause Maîtriser suffisamment les principes fondamentaux du logiciel
  pour être en mesure de se perfectionner par la suite.
\end{enumerate}

\bigskip \pause \strong{Principes pédagogiques}

\begin{enumerate}
\def\labelenumi{\arabic{enumi}.}
\item
  \pause Pratique permanente et orientée métier;
\item
  \pause Autonomie dans l'apprentissage et progression à son rythme;
\item
  \pause Accompagnement personnalisé par le formateur.
\end{enumerate}

\end{frame}

\begin{frame}[fragile]{Supports (1) : pages web de la formation}

Le support principal de la formation est un ensemble de pages web
accessibles à l'adresse \href{http://r.slmc.fr}{\link{r.slmc.fr}}.

\bigskip \pause Ces pages web contiennent l'ensemble du contenu de la
formation:

\begin{itemize}
\tightlist
\item
  présentation, explication et illustration des notions avec des
  exemples de code;
\item
  cas pratiques corrigés.
\end{itemize}

\bigskip \pause \strong{Remarques}

\begin{itemize}
\tightlist
\item
  Ces pages web peuvent être sauvegardées sous forme de fichiers
  \texttt{.html} et consultées hors connexion;
\item
  Les supports figurent dans le répertoire de la formation sur AUS :
  \texttt{Y:\textbackslash{}Documentation\textbackslash{}R\textbackslash{}R\_initiation\textbackslash{}}.
\end{itemize}

\end{frame}

\begin{frame}{Supports (2) : livret et présentations}

Le contenu de la formation est également disponible dans un
\strong{livret imprimé}:

\begin{itemize}
\tightlist
\item
  faciliter l'appropriation des notions;
\item
  réduire la fatigue visuelle;
\item
  constituer une référence à l'issue de la formation.
\end{itemize}

\bigskip \pause De \strong{courtes présentations} ponctuent la
formation:

\begin{itemize}
\tightlist
\item
  rythmer la progression;
\item
  présenter les objectifs des modules;
\item
  insister sur les notions les plus importantes.
\end{itemize}

\bigskip \pause \strong{Remarque} Tous les supports de la formation ont
été produits depuis R avec
\href{http://rmarkdown.rstudio.com/}{\link{R Markdown}} (\emph{cf.}
\href{http://t.slmc.fr/perf}{\link{R perfectionnement}}).

\end{frame}

\begin{frame}{Progression}

La formation est articulée autour de \textbf{trois modules}:

\begin{enumerate}
\def\labelenumi{\arabic{enumi}.}
\item
  \pause \textbf{Prise en main du logiciel}: se repérer dans l'interface
  et savoir explorer des données;
\item
  \pause \textbf{Manipuler les éléments fondamentaux du langage}:
  acquérir une connaissance solide des briques élémentaires de R;
\item
  \pause \textbf{Travailler avec des données statistiques}: articuler
  les briques du module 2 pour la statistique appliquée.
\end{enumerate}

\bigskip \pause \strong{Articulation générale}

\begin{itemize}
\tightlist
\item
  \pause Les modules 1 et 3 sont \textbf{orientés métier}: travail sur
  des données Insee dans une perspective \enquote{chargé d'études}.
\item
  \pause Le module 2 est un \textbf{détour nécessaire} pour maîtriser
  les manipulations effectuées dans le module 3.
\end{itemize}

\end{frame}

\begin{frame}{Organisation pratique}

\strong{Travail en autonomie}

\begin{itemize}
\tightlist
\item
  \vspace{-0.2cm} partie \enquote{cours} à partir du livret;
\item
  cas pratiques corrigés sur AUS;
\item
  \pause \textbf{vous m'appelez en cas de difficulté}.
\end{itemize}

\bigskip \pause \strong{Horaires}

\begin{itemize}
\tightlist
\item
  \vspace{-0.2cm} proposition : 9h30-12h20 puis 13h40-16h30;
\item
  pauses : 11h, 15h et quand vous sentez que c'est nécessaire !
\end{itemize}

\bigskip \pause \strong{Conditions de travail} La formation est
\textbf{intensive}:

\begin{itemize}
\tightlist
\item
  \vspace{-0.2cm} prenez le temps de bien paramétrer votre espace de
  travail (éclairage, siège, etc.) et faites des pauses régulièrement;
\item
  n'hésitez pas à m'indiquer tout ce qui peut améliorer votre confort.
\end{itemize}

\end{frame}

\section{Module 1 : Prise en main du
logiciel}\label{module-1-prise-en-main-du-logiciel}

\subsection*{Prise en main du logiciel}\label{prise-en-main-du-logiciel}
\addcontentsline{toc}{subsection}{Prise en main du logiciel}

\begin{frame}{Objectifs et organisation}

\strong{Objectifs}

\begin{enumerate}
\def\labelenumi{\arabic{enumi}.}
\tightlist
\item
  Acquérir des points de repère dans l'interface de R;
\item
  Mener quelques traitements simples pour observer le fonctionnement du
  logiciel;
\item
  Introduire des problématiques métier: travail sur des données,
  importation de fichiers de données SAS.
\end{enumerate}

\bigskip \pause \strong{Organisation}

\begin{enumerate}
\def\labelenumi{\arabic{enumi}.}
\item
  Un peu d'histoire et quelques grands principes
\item
  Découverte de l'interface
\item
  Charger et explorer des données
\item
  Importer des données à l'aide de \emph{packages}
\end{enumerate}

\end{frame}

\begin{frame}[fragile]{\large Un peu d'histoire et quelques grands
principes}

Insister sur les \textbf{spécificités de R}, notamment par rapport à
SAS:

\begin{itemize}
\tightlist
\item
  R est sensible à la casse;
\item
  dans R, les chemins doivent être indiqués avec des \texttt{/} et non
  des \texttt{\textbackslash{}}.
\end{itemize}

\bigskip \pause Plus généralement, R s'apparente davantage à un
\textbf{langage de programmation \enquote{classique}} (Python par
exemple):

\begin{quote}
\emph{To understand computations in R, two slogans are helpful:}

\begin{itemize}
\item
  \emph{Everything that exists is an object.}
\item
  \emph{Everything that happens is a function call.}
\end{itemize}

\emph{John Chambers}
\end{quote}

\end{frame}

\begin{frame}{Découverte de l'interface}

Faire de \textbf{premières manipulations} dans les deux interfaces de R
disponibles sur AUS:

\begin{itemize}
\item
  R \enquote{classique}: programme très dépouillé, essentiellement
  utilisé en mode \enquote{console};
\item
  Rstudio: environnement de développement intégré qui facilite
  considérablement l'\textbf{écriture de scripts} (colorisation du code,
  auto-complétion, etc.).
\end{itemize}

\bigskip \pause Acquérir un \textbf{vocabulaire de base}:

\begin{itemize}
\tightlist
\item
  création d'objets simples et opérations arithmétiques;
\item
  affichage et manipulation des objets stockés en mémoire;
\item
  utilisation de l'aide intégrée dans le logiciel;
\item
  écriture d'une première fonction personnalisée.
\end{itemize}

\end{frame}

\begin{frame}[fragile]{Charger et explorer des données}

Savoir \textbf{utiliser des données} dans R :

\begin{itemize}
\tightlist
\item
  chargement d'un fichier de données \texttt{.RData};
\item
  principales caractéristiques des données : nombre d'observations,
  affichage des premières lignes, etc.
\end{itemize}

\bigskip \pause Mener des \textbf{traitements simples} avec R :

\begin{itemize}
\tightlist
\item
  indicateurs statistiques usuels: moyenne d'une variable quantitative,
  distribution d'une variable qualitative;
\item
  ventilation des traitements selon les modalités d'une variable
  qualitative;
\item
  production de graphiques simples.
\end{itemize}

\end{frame}

\begin{frame}[fragile]{Importer des données à l'aide de \emph{packages}}

\textbf{Importer des données} qui ne sont pas en format R natif:

\begin{itemize}
\tightlist
\item
  fichiers plats: \texttt{.txt}, \texttt{.csv}, \texttt{.dlm};
\item
  fichiers SAS: \texttt{.sas7bdat}.
\end{itemize}

\bigskip \pause \textbf{Exporter des données} au format R natif.

\bigskip \pause Percevoir l'\textbf{importance des \emph{packages}} dans
R :

\begin{itemize}
\tightlist
\item
  installation;
\item
  chargement pour accéder à de nouvelles fonctions.
\end{itemize}

\end{frame}

\section{Module 2 : Manipuler les éléments fondamentaux du
langage}\label{module-2-manipuler-les-elements-fondamentaux-du-langage}

\subsection*{Manipuler les éléments fondamentaux du
langage}\label{manipuler-les-elements-fondamentaux-du-langage}
\addcontentsline{toc}{subsection}{Manipuler les éléments fondamentaux du
langage}

\begin{frame}{Objectifs et organisation}

\strong{Objectifs}

\begin{enumerate}
\def\labelenumi{\arabic{enumi}.}
\tightlist
\item
  Introduire progressivement les briques élémentaires du langage de R;
\item
  Connaître leurs propriétés et savoir les manipuler;
\item
  Enrichir son vocabulaire de fonctions.
\end{enumerate}

\bigskip \pause \strong{Organisation}

\begin{enumerate}
\def\labelenumi{\arabic{enumi}.}
\item
  Manipuler les vecteurs
\item
  Manipuler les matrices
\item
  Manipuler les listes
\end{enumerate}

\end{frame}

\begin{frame}{Manipuler les vecteurs}

Les vecteurs sont les éléments fondamentaux du langage de R. Ils servent
notamment :

\begin{itemize}
\tightlist
\item
  à coder l'information statistique: les variables d'une table sont des
  vecteurs;
\item
  à modifier le contenu d'une table: créer de nouvelles variables,
  sélectionner des variables et des observations;
\item
  à calculer des indicateurs statistiques: les \emph{inputs} de la
  plupart des fonctions statistiques sont des vecteurs.
\end{itemize}

\pause \bigskip \strong{Progression de la partie}

\begin{enumerate}
\def\labelenumi{\arabic{enumi}.}
\tightlist
\item
  Création de vecteurs et sélection d'éléments;
\item
  Spécificités des différents types de vecteurs et manipulations;
\item
  Compléments : valeurs spéciales, conversion de type
\end{enumerate}

\end{frame}

\begin{frame}[fragile]{Manipuler les matrices}

Les matrices sont une généralisation directe des vecteurs en deux
dimensions (ou plus, on parle alors de \texttt{array}).

\pause Cette structure à deux dimensions les rapproche par certains
égards des tableaux de données statistiques.

\pause \bigskip \strong{Objectif de la partie} Création de matrices et
sélection d'éléments.

\end{frame}

\begin{frame}{Manipuler les listes}

Les listes sont des objets plus complexes que les vecteurs ou les
matrices.

\pause En particulier, elles peuvent contenir des \textbf{éléments de
types différents} (numérique, caractère, logique, etc.), voire d'autres
listes.

\pause Cela en fait un type d'objet particulièrement souple pour
\textbf{stocker et exploiter une information riche et structurée}.

\aparte{Exemples}{Résultats d'un modèle de régression ou d'une méthode de classification.}

\pause \bigskip \strong{Progression de la partie}

\begin{enumerate}
\def\labelenumi{\arabic{enumi}.}
\tightlist
\item
  Création de listes et sélection d'éléments;
\item
  Calculs sur les listes.
\end{enumerate}

\end{frame}

\section{Module 3 : Travailler avec des données
statistiques}\label{module-3-travailler-avec-des-donnees-statistiques}

\subsection*{Travailler avec des données
statistiques}\label{travailler-avec-des-donnees-statistiques}
\addcontentsline{toc}{subsection}{Travailler avec des données
statistiques}

\begin{frame}[fragile]{Objectifs et organisation}

\strong{Objectifs}

\begin{enumerate}
\def\labelenumi{\arabic{enumi}.}
\tightlist
\item
  Revenir à des problématiques métiers courantes : sélection
  d'observations et de variables, tri d'une table, etc.;
\item
  Mobiliser les fonctions abordées dans le module 2;
\item
  Présenter le calcul de statistiques descriptives avec R.
\end{enumerate}

\bigskip \pause \strong{Organisation}

\begin{enumerate}
\def\labelenumi{\arabic{enumi}.}
\item
  Manipuler les \texttt{data.frame}
\item
  Calculer des statistiques descriptives
\item
  Quelques liens pour aller plus loin
\end{enumerate}

\end{frame}

\begin{frame}[fragile]{Manipuler les \texttt{data.frame}}

Le type \texttt{data.frame} est le type le plus souvent utilisé pour
exploiter des données statistiques.

\pause Il s'agit d'un \textbf{cas particulier de listes} qui
\textbf{partage beaucoup de propriétés avec les matrices}.

\bigskip \pause \strong{Progression}

\begin{enumerate}
\def\labelenumi{\arabic{enumi}.}
\item
  Création d'un \texttt{data.frame} et sélection d'éléments : sélection
  d'observations et de variables;
\item
  \pause Création ou modification de variables dans un
  \texttt{data.frame};
\item
  \pause Modification de la structure d'un \texttt{data.frame} : tri,
  concaténation, fusions, etc.;
\item
  \pause Calculs sur un \texttt{data.frame} : application d'une fonction
  à toutes les variables, application d'une fonction par groupe.
\end{enumerate}

\end{frame}

\begin{frame}{Calculer des statistiques descriptives}

Plusieurs fonctions permettant d'effectuer des statistiques descriptives
sont introduites dans les modules 1 et 2.

\pause Cette partie présente ces fonctions de façon \textbf{plus
systématique}, notamment autour de la question des \textbf{statistiques
descriptives pondérées}.

\bigskip \pause \strong{Progression}

\begin{enumerate}
\def\labelenumi{\arabic{enumi}.}
\item
  Statistiques descriptives sur variables quantitatives;
\item
  Statistiques descriptives sur variables qualitatives;
\item
  Création et paramétrisation de graphiques;
\item
  \pause Application à l'enquête Pisa 2012.
\end{enumerate}

\end{frame}

\begin{frame}[fragile]{Quelques liens pour aller plus loin}

Les méthodes statistiques plus avancées sortent du cadre de cette
formation au logiciel R.

\pause Des liens sont néanmoins fournis pour approfondir deux aspects
souvent utiles en pratique :

\begin{itemize}
\tightlist
\item
  \textbf{analyse de données multidimensionnelle} avec le
  \href{https://CRAN.R-project.org/package=FactoMineR}{\link{\textit{package} \texttt{FactoMineR}}};
\item
  \href{http://teaching.slmc.fr/mqs2/index.html}{\link{\textbf{modèles de régression}}}
  avec les fonctions \texttt{lm()} et \texttt{glm()}.
\end{itemize}

\bigskip \pause Un lien vers la
\href{http://teaching.slmc.fr/perf/index.html}{\textbf{\link{formation R perfectionnement}}}
est également proposé:

\begin{itemize}
\tightlist
\item
  outils et méthodes pour se perfectionner en R;
\item
  traitements avancés sur des données dans R;
\item
  graphiques et \emph{reporting} avec R.
\end{itemize}

\end{frame}

\end{document}
